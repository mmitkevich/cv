\documentclass[10pt]{beamer}

\usetheme[progressbar=frametitle]{metropolis}
\usepackage{appendixnumberbeamer}
\usepackage{booktabs}
\usepackage[scale=2]{ccicons}
\usepackage{pgfplots}
\usepgfplotslibrary{dateplot}
\usepackage{listings}
\usepackage{xspace}
\newcommand{\themename}{\textbf{\textsc{metropolis}}\xspace}

\title{Mikhail Mitkevich}
\subtitle{R/C++ developer}
\date{\today}
%\date{}
%\author{Matthias Vogelgesang}
%\institute{Center for modern beamer themes}
% \titlegraphic{\hfill\includegraphics[height=1.5cm]{logo.pdf}}

\begin{document}

\maketitle

\begin{frame}[fragile]{Outline}

\begin{itemize}
\item  \textit{1994} Learned C++ by Stroustrup, aged 15
\item  \textit{1996-2002} Msc in applied mathematics, \href{https://www.timeshighereducation.com/world-university-rankings/moscow-institute-physics-and-technology}{Moscow institute of physics and technology (MIPT.ru)} 
\item \textit{2002-2006} C++/win32 developer, custom DSP chip gcc toolchain porting, \href{http://www.ispras.ru/en/}{ispras.ru} 
\item \textit{2006-2008} C++/C\# developer, reinforcement learning applied to portfolio optimization, \href{http://www.ccas.ru/index-e.htm}{ccas.ru}
\item  \textit{2008-2016} java/linux developer, realtime distributed call center software, java/linux/oracle,  \href{https://en.wikipedia.org/wiki/Alcatel-Lucent}{Alcatel-Lucent}
\item  \textit{2016...} R/C++ and java developer in a proprietary hedge fund, both research and low latency execution roles
\end{itemize}
\end{frame}

\begin{frame}{Programming}
  \begin{columns}[T,onlytextwidth]
    \column{0.33\textwidth}
      OS
      \begin{itemize}
        \item Ubuntu/CentOS
        \item OSX 
        \item Windows NT (in the past)
      \end{itemize}

    \column{0.33\textwidth}
      Languages
      \begin{itemize}
        \item C++ 17
        \item java 8
        \item R, python
        \item javascript/nodejs
        \item C\# (in the past)
      \end{itemize}

    \column{0.33\textwidth}
      Databases
      \begin{itemize}
        \item postgreSQL
        \item clickhouse
        \item lmdb, libsqlite
        \item oracle, mysql (in the past)
      \end{itemize}
  \end{columns}
\end{frame}


\begin{frame}{Tools and libraries}
\begin{columns}[T,onlytextwidth]
	\column{0.5\textwidth}
	Tools 
	\begin{itemize}
		\item cmake, make
		\item gdb, varlgrind
		\item conan, bintray.com
		\item gradle, maven
		\item git, gitlab, github
	\end{itemize}	
	\column{0.5\textwidth}
	Libraries
	\begin{itemize}
		\item C++: stl, boost \{asio, beast, thread, filesystem\}, Rcpp, rxcpp
		\item R : dplyr, ggplot, purrr, shiny
		\item python: pandas, numpy, asyncio
	\end{itemize}
\end{columns}
\end{frame}

\begin{frame}{My code: Rticks backtesting library (C++)}
\begin{itemize}
	\item gamma strategy (mean reversion): \href{https://github.com/mmitkevich/rticks_v1/blob/master/src/gamma.h} {https://github.com/mmitkevich/rticks\_v1/
		blob/master/src/gamma.h}
	\item functional reactive streams design inspired by  Reactive Extensions library (reactivex.io) \href{http://rxmarbles.com}{http://rxmarbles.com}, \href{http://reactivex.io}{http://reactivex.io}, \href{https://github.com/ReactiveX/RxCpp/tree/master/Rx/v2/src/rxcpp}{http://github.com/ReactiveX/RxCpp}
	\item utilizes C++ 14 and templates to produce optimized code using higher-level concepts
	\item Goals: to became a realtime stream-based alternative to R's dplyr, suitable for backtest/optimization and low-latency exectuion
\end{itemize}
\end{frame}

\begin{frame}{Use case: crude marketmaking model 1/2}
Suppose that:
\begin{itemize}
	\item We want to be market makers of closest quarterly Crude Oil futures contracts on Comex and Nymex exchanges
	\item And keep our portfolio delta $\Delta(t)$ (e.g. normalized net futures position expressed in dollars) small enough to keep risks small:
	$$
	\Delta(t)=[\sum_{i\in \{CL.1@COMEX,CL.1@NYMEX\}} \Delta_{i}(t) ] \rightarrow \min_{STRATEGY.PARAMETERS}
	$$
	\item To start with, we choose our bid price $S_{BID}$ and ask price $S_{ASK}$ at fixed distance $D_{HALF.SPREAD}$ from some fair price $S_{FAIR.PRICE}$, defined later:
	$$
	S_{BID,ASK} = S_{FAIR.PRICE}  \mp D_{HALF.SPREAD} 
	$$
	\item For simplicity we treat both futures essentially the same so we could mix their order books into single order book and use midprice as fair price:
	$$
	S_{FAIR.PRICE} = 0.5(S_{BEST.BID}+S_{BEST.ASK})
	$$
\end{itemize}
\end{frame}

\begin{frame}{Use case: crude marketmaking model 2/2}
\begin{itemize}
\item We choose our buy order quantity  $\Gamma_{BUY}$ and (negative) sell order quantity $\Gamma_{SELL}$ to keep our Delta small, but continue to  make markets:
$$\Gamma^i_{BUY}(t)=\Gamma_0+\max(0,-\Delta) $$
$$\Gamma^i_{SELL}(t)=-(\Gamma_0+\max(0,\Delta)) $$
\item When executed, these orders would bring our $\Delta$ closer to zero
\end{itemize}
\end{frame}

\begin{frame}[fragile]{Concept: Functional strategy DSL 0}
\begin{itemize}
\item In the following couple of slides I describe preliminary design of functional DSL language aimed for R \& python quants to formally specify
high frequency trading strategies for backtesting purposes
\item Strategies are formulated as functional transformations of data and execution feeds (streams) so that the same code could be used in production as well
\end{itemize}
\end{frame}

\begin{frame}[fragile]{Concept: Market data feeds 1}
First, note that conceptually all level1, level2, level3 and execution report feeds generalize to streams of tuples
\begin{lstlisting}
feed(t) = [timestamp, symbol, op, price, qty]
\end{lstlisting}
Where operation $op$ could have following values:
\begin{itemize}
	\item PLACED: order was inserted into the order book
	\item CANCELED: order was removed from the order book
	\item FILLED: order was fully filled and is out of the book
	\item PART\_FILLED: order filled partially and is still in the book
	\item UPDATED: cumulative quantity at order book price level has changed
	\item RESET, SNAPSHOT: these two could be used to synchronize order book snapshots with server 
\end{itemize} 
We leave transactions (PLACE, AMEND, CANCEL) out of scope for now. 
\end{frame}

\begin{frame}[fragile]{Concept: streams DSL inspired by GNU R's dplyr package 2}
\href{dplyr}{dplyr} is a popular GNU R package for data.frame manipulations.

Let's imagine dplyr-like DSL to define our trading strategy. Let's start with defining the feeds using hypothetical $data\_streams$ function
\begin{lstlisting}
   symbols = c("CL.1@COMEX", "CL.1@NYMEX")
   s0 = data_streams(
     level2=level2_feed(symbols),
     execs=exec_feed("mike@GS","peter@DB"))
\end{lstlisting}
So printing feeds could give us the following output
\begin{lstlisting}
>tail(feeds)
level2|UPDATE|17:23:14.323|CL.1@COMEX | 65.50 | 100
level2|UPDATE|17:23:14.324|CL.1@NYMEX | 65.52 | -100
execs |FILL  |17:23:14.324|CL.1@COMEX | 65.51 | 50 | mike@GS
\end{lstlisting}
\end{frame}
\begin{frame}[fragile]{Concept: stream mutation in functional style 3}
Actual position could be calculating by summing $qty$ for all $executions$:
\begin{lstlisting}
   s1 = s0 %>% mutate(
     position=cumsum(executions))
\end{lstlisting}
Note, that:
\begin{itemize}
	\item summation is done for each symbol individually.
	\item logically position stream is also tuple [price, qty] so we should define arithmetic operations on such tuples
\end{itemize}
\end{frame}

\begin{frame}[fragile]{Concept: simplify arithmetics}

Let's denote 
\begin{itemize}
	\item $S$:price (in USD), 
	\item $Q$:qty (in items), 
	\item $X=[S,Q]$:vector of both,
	\item $U(X)=SQ$: value of this vector
\end{itemize}

Let's define plus, minus, mul, div operators
$$X_1+X_2=[S_1,Q_1]+[S_2,Q_2]=[S_1Q_1+S_2Q_2, 1] \rightarrow
U=S_1Q_1+S_2Q_2=U_1+U_2$$
$$X_1-X_2=[S_1Q_1-S_2Q_2, 1] \rightarrow U=U_1-U_2$$
$$X_1*X_2=[S_1Q_1S_2Q_2,1]\rightarrow U=U_1U_2$$
$$X_1/X_2=[S_1Q_1/S_2Q_2,1]\rightarrow U=U_1/U_2$$

This corresponds to casting [price, qty] tuple to price*qty real number, and casting back real number to [number, 1] tuple.
This would simplify arithmetics
\end{frame}

\begin{frame}[fragile]{Concept: mixing order books 4}
To mix order books we transform $level2$ feed to have common symbol and then
use $level2\_to\_level1$ function which
\begin{itemize}
	\item takes $level2$ stream as its input
	\item reconstructs order book state on it (filtering matched price levels)
	\item outputs required $level1$ feed (which basically contains only SNAPSHOT records):
\end{itemize}
\end{frame}

\begin{frame}[fragile]{Concept: calculating required values 5}
Also we transform last $position$ and $mid$  price into delta.
\begin{lstlisting}
   contract_lot=10 # barrels
   s2 = s1 %>% mutate(
     level2=level2%>%mutate(symbol='CL.1@MIXED'),
     level1=level2_to_level1(level2),
     best.bid=level1 %>% filter(qty>0),
     best.ask=level1 %>% filter(qty<0),
     fair.price=0.5*(best.bid+best.ask),
     delta=position*fair.price*contract_lot)
\end{lstlisting}

Note, that stream of position is a single number,
\end{frame}

\begin{frame}[fragile]{Concept: reducing over portfolio 6}
In the previous example we have calculated delta for each contract, so now we need to summarize it over all symbols in the portfolio
\begin{lstlisting}
  s3 = s2 %>% mutate(
    Delta=reduce(delta, ~ .x+.y)
  )
\end{lstlisting}
Now our $s3$ stream contains everything we need to calculate fair price and our buy/sell quotes:
\begin{itemize}
	\item Delta=$\Delta$: total delta of the portfolio, in dollars
	\item best.bid=$S_{BID}$, best.ask=$S_{ASK}$, fair.price=$S_{FAIR.PRICE}$: mixed book bid, ask and fair price calculated as their average
\end{itemize}
\end{frame}

\begin{frame}[fragile]{Concept: calculating final market maker quotes 7}
\begin{lstlisting}
  gamma0=100 # USD
  half_spread=0.01 # USD
  s4 = s3 %>% mutate(
     buy = (fair.price-half_spread) %>% mutate(
          qty=(max(0,-Delta)+gamma0)
                 /contract_lot/price),
     sell = (fair.price+half_spread) %>% mutate(
          qty=-(max(0,Delta)+gamma0)
                /contract_lot/price))
\end{lstlisting}
These buy and ask quotes are subject to be maintained in the market by high-frequency execution service. This service receives desired buy and sell quotes (with respective quantities) and sends orders accordingly.
\end{frame}

\begin{frame}{Concept: Execution, further improvements 8}
\begin{itemize}
\item Possible latency reduction could be achieved by mixed software/hardware design. For example, our simple strategy only uses simple arithmetics, max/min operations and orderbook mixing operations. If properly implemented in FPGA hardware it could be possible to reduce latency introduced by software.
\item DSL language could be implemented not only with R dplyr-like syntax. Actually all 'mutate' clauses are really optional. Usability by quants is the main concern.
\end{itemize}
\end{frame}

\begin{frame}{My code: offheap order matching engine (Java)}
\begin{itemize}
	\item \href{https://github.com/mmitkevich/lob-java/blob/master/src/main/java/org/freeticks/lob/OffHeapBook.java}
	{https://github.com/mmitkevich/lob-java/blob/master/src/main/java/
		org/freeticks/lob/OffHeapBook.java}
	\item uses DirectBuffer  to manipulate offheap memory and minimize allocations
	\item price level index implemented as plain array instead of sorted rb-tree
	\item orders are allocated in offheap circular arena
\end{itemize}
\end{frame}

\begin{frame}[fragile]{My code: lockless SPSC queue (.NET)}
\begin{itemize}
	\item \href{https://github.com/mmitkevich/Snail/blob/master/Snail/Threading/BQueue.cs} {https://github.com/mmitkevich/Snail/
		blob/master/Snail/Threading/BQueue.cs}
	\item CLR implementation of bounded lockless queue addressing false sharing problem for low latency core-to-core communication
\end{itemize}
\end{frame}

\begin{frame}{Links}

 	 \begin{center}\href{https://linkedin.com/in/mmitkevich}{https://linkedin.com/in/mmitkevich}\end{center}
 	 \begin{center}\href{https://github.com/mmitkevich}{https://github.com/mmitkevich}\end{center}

\end{frame}
\end{document}




